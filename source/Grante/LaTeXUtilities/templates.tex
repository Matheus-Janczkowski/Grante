\documentclass[12pt,3p, twoside]{elsarticle}

% Calls our own package, which already loads a bunch of other packages
\usepackage{LaTeXUtilities}

% Sets the left, right, top, and bottom margins, respectively
\SetMargins{3.0cm}{2.0cm}{3.0cm}{2.0cm}

% Sets the left, right, top, and bottom margins, respectively. Odd and even numbered-pages will have alternated margins
%\SetBindingMargins{3.0cm}{2.0cm}{3.0cm}{2.0cm}

% Sets the ABNT spacings. Use argument FrenteVerso if you want the margins to be alternating from odd to even numbered pages. Comment this line or delete it if you don't want ABNT spacings
%\SetABNTSpacings{FrenteVerso}

% Takes out the tag "Preprint submitted to Elsevier"
\TakeOutPreprintElsevier

\begin{document}

\begin{frontmatter}

\title{Templates for figures, tables, and other structures}

\affiliation[1]{department={Mechanical Engineering Department,},
organization={Santa Catarina Federal University},
addressline={Eng. Agronômico Andrei Cristian Ferreira Street, no number - Trindade},
city={Florianópolis},
country={Brazil}}

\affiliation[2]{department={Mechanical Engineering Department,},
organization={Santa Catarina State University},
addressline={Paulo Malschitzki Street, 200 - Zona Industrial Norte},
city={Joinville},
country={Brazil}}

\author[1]{Matheus Janczkowski}
\author[2]{Author 2}
\author[1]{Author 3}

\begin{abstract}  
Arma virumque canō, Trōiae quī prīmus ab ōrīs
Ītaliam, fātō profugus, Lāvīniaque vēnit
lītora — multum ille et terrīs iactātus et altō
vī superum saevae memorem Iūnōnis ob īram;
multa quoque et bellō passus, dum conderet urbem,
inferretque deōs Latiō — genus unde Latīnum,
Albānīque patrēs, atque altae moenia Rōmae.
\end{abstract}


\begin{keyword}
C. Iulius Caesar \sep Germanicus Iulius Caesar \sep Cn. Iulius Agricola \sep M. Ulpius Traianus
\end{keyword}

\end{frontmatter}

% Adds the page of contents
\ContentsPage

\section{Equation}\label{sec:equation}

Hactenus principia tradidi, quibus philosophiæ naturalis partes mathematicae innituntur; ex his propositiones sequenti libro deducentur,
%
\begin{flalign}\label{eq:segunda_lei}
\boldsymbol{F}=m\boldsymbol{a},
\end{flalign}
%
where $\boldsymbol{F}$ is the force vector; $m$ is the mass; and $\boldsymbol{a}$ is the acceleration vector.

Equation \eqref{eq:segunda_lei} is the cornerstone of classical mechanics. Even in continuum mechanics, equilibrium equations are derived using Eq. \eqref{eq:segunda_lei}. A quadratic form in matrix notation is
%
\begin{flalign}\label{eq:matrix_quadratic_form}
\phi=\Transpose{\mathbf{a}}\mathbf{D}\mathbf{a}.
\end{flalign}

\subsection{Citation}\label{subsec:citation}

There are two kinds of citation: one that is not the phrase's subject \cite{BoyceDiPrima}; another where \citet{BoyceDiPrima} are the subject indeed.

\section{Figure}\label{sec:figure}

\begin{figure}[h!]%
\centering
\includegraphics[width=0.6\textwidth]{Figures/multiscale_figure.png}
\caption{Multiscale applications.}
\label{fig:multiscale_applications}
\end{figure}

\section{Table}\label{sec:table}

\begin{table}[h]
\centering
\begin{tabular}{|p{1.5cm}|p{1.5cm}|p{1.5cm}|}
%\begin{tabular}{ C{1.7cm} |  C{2cm} | C{2cm} | C{1.5cm} | C{1cm} | C{2cm} | C{2cm}}
\hline
\cellcolor{LightGray}{Case} & 
\cellcolor{LightGray}{$N$} & \cellcolor{LightGray}{$l_{b}\;[mm]$} \\
\hline
\cellcolor{LightGray}{$\text{I}_{\text{bending}}$} & 0.001 & 1.0\\
\hline
\cellcolor{LightGray}{$\text{II}_{\text{bending}}$} & 0.08 & 1.0 \\
\hline
\end{tabular}
\caption{$N$ and $l_{b}$ for the bending case. The same parameter values were assigned to both matrix and fiber phases. The same parameters are set for the DNS and for the RVE.}%
\label{tab:material_parameters_bending}
\vspace{2mm}
\end{table}

\subsection{Schedule}\label{subsec:schedule}

\begin{table}[!htbp]
\centering
\begin{tabular}{|p{2cm}|p{1.25cm}|p{1.25cm}|p{1.25cm}|p{1.25cm}|p{1.25cm}|p{1.25cm}|p{1.25cm}|p{1.25cm}|}
%\begin{tabular}{ C{1.7cm} |  C{2cm} | C{2cm} | C{1.5cm} | C{1cm} | C{2cm} | C{2cm}}
\hline
\cellcolor{LightGray} & \multicolumn{4}{|c|}{\cellcolor{LightGray}{2024}} & \multicolumn{4}{|c|}{\cellcolor{LightGray}{2025}} \\
\cline{2-9}
\multirow{-2}{*}{\cellcolor{LightGray}{Atividades}} & \cellcolor{LightGray}{1\oOrdinal$\;$Bi} & \cellcolor{LightGray}{2\oOrdinal $\;$Bi} & \cellcolor{LightGray}{3\oOrdinal $\;$Bi} & \cellcolor{LightGray}{4\oOrdinal $\;$Bi} & \cellcolor{LightGray}{1\oOrdinal $\;$Bi} & \cellcolor{LightGray}{2\oOrdinal $\;$Bi} & \cellcolor{LightGray}{3\oOrdinal $\;$Bi} & \cellcolor{LightGray}{4\oOrdinal $\;$Bi} \\
\hline
01 &  \cellcolor{DarkGray} & \cellcolor{DarkGray} & \cellcolor{DarkGray} & \cellcolor{DarkGray} &  &  &  & \\
\hline
02 & \cellcolor{DarkGray} &  &  &  &  &  &  & \\
\hline
03 &  & \cellcolor{DarkGray} &  &  &  &  &  & \\
\hline
04 &  &  & \cellcolor{DarkGray} & \cellcolor{DarkGray} &  &  &  & \\
\hline
05 & \cellcolor{DarkGray} &  &  &  &  &  &  & \\
\hline
06 &  & \cellcolor{DarkGray} &  &  &  &  &  & \\
\hline
07 &  &  & \cellcolor{DarkGray} &  &  &  &  & \\
\hline
08 &  &  &  & \cellcolor{DarkGray} & \cellcolor{DarkGray} &  &  & \\
\hline
09 &  &  &  &  & \cellcolor{DarkGray} &  &  & \\
\hline
10 &  &  &  &  & \cellcolor{DarkGray} &  &  & \\
\hline
11 &  &  &  &  & \cellcolor{DarkGray} & \cellcolor{DarkGray} &  & \\
\hline
12 &  &  &  &  &  & \cellcolor{DarkGray} & \cellcolor{DarkGray} & \\
\hline
13 &  &  &  &  &  &  & \cellcolor{DarkGray} & \cellcolor{DarkGray}\\
\hline
14 &  &  &  &  &  &  & \cellcolor{DarkGray} & \cellcolor{DarkGray}\\
\hline
15 &  &  &  &  &  &  &  & \cellcolor{DarkGray}\\
\hline
16 &  &  &  &  &  &  &  & \cellcolor{DarkGray}\\
\hline
30 &  &  &  & \cellcolor{DarkGray} & \cellcolor{DarkGray} & \cellcolor{DarkGray} & \cellcolor{DarkGray} & \\
\hline
\end{tabular}
\caption{Plano de trabalho - atividades planejadas para cada bimestre (Bi) dos anos de $2024$ e de $2025$.}%
\label{tab:plano_de_trabalho_2024_2025}
\vspace{2mm}
\end{table}

\subsection{Enumeration}\label{subsec:enumeration}

\begin{enumerate}
\item Item 1;
\item Item 2.
\end{enumerate}

\section{Flowchart}\label{sec:flowchart}

\begin{figure}[!htb]
\centering
\begin{tikzpicture}
\node [terminator, fill=red!20] at (0,0) (start) {\textbf{Problem}};
\node [process, fill=blue!20] at (0,-1.5) (process1) {Analyze and understand the problem};
\node [process, fill=blue!20, text width=45mm] at (0,-3) (process2) {Define steps to solve it and create a flowchart};
\node [process, fill=blue!20] at (0,-4.5) (process3) {Write a pseudocode (PC)};
\node [decision, fill=yellow!40, text width=20mm] at (0,-7) (decision) {Does the PC make sense?};
\node [process, fill=blue!20, text width=40mm] at (5.7,-7) (implementacao) {Implement it in a programming language};
\node [decision, fill=yellow!40, text width=15mm] at (5.7,-10.7) (decision2) {Does it work?};
\node [process, fill=blue!20, text width=25mm] at (2.2,-10.7) (teste) {Run several different tests};
\node [decision, fill=yellow!40, text width=15mm] at (-1.3,-10.7) (decision3) {Does it work?};
\node [process, fill=green!20, text width=15mm] at (-1.3,-12.9) (sucesso) {Success};
\node [terminator, fill=blue!45] at (2.2,-12.9) (end) {\textbf{Solution}};

\node[draw=none] at (-2.3, -6.75) {No};
\node[draw=none] at (2.3, -6.75) {Yes};
\node[draw=none] at (7.5, -10.45) {No};
\node[draw=none] at (4.1, -10.45) {Yes};
\node[draw=none] at (-0.6, -9.05) {No};
\node[draw=none] at (-0.9, -12.1) {Yes};

\path [connector] (start) -- (process1);
\path [connector] (process1) -- (process2);
\path [connector] (process2) -- (process3);
\path [connector] (process3) -- (decision);
\draw[-latex'] (decision) -- (-3,-7) -- (-3,-3) -- (process2);
\path [connector] (decision) -- (implementacao);
\path [connector] (implementacao) -- (decision2);
\draw[-latex'] (decision2) -- (8.3,-10.7) -- (8.3,-4.5) -- (process3);
\path [connector] (decision2) -- (teste);
\path [connector] (teste) -- (decision3);
\draw[-latex'] (decision3) -- (-1.3,-9.3) -- (3,-9.3) -- (implementacao);
\path [connector] (decision3) -- (sucesso);
\path [connector] (sucesso) -- (end);
\end{tikzpicture}
\caption{Flowchart illustrating how to program.}
\label{fig:flowchart_programming}
\end{figure}


\section{Algorithm}\label{sec:algorithm}\begin{algorithm2e}[ht!]
\caption{Generation of plane truss meshes to simulate the transient response due to periodic excitation. Comparisons between the HS method, the Newmark-beta method and the analytical response using the Generalized Integrating Factor will be made} \label{alg:efficient_evaluation_HS}
\begin{algorithmic}
\State Sets the initial problem with number of nodes to determine the edges of the structure and their positions\\
Sets the excited degrees of freedom and the function of the force\\
Defines a set of divisions of the domain in the horizontal direction $x$, $\lbrace 1, 2, 3, \dots, n_{x}\rbrace$\\
Defines a set of divisions of the domain in the vertical direction $y$, $\lbrace 1, 2, 3, \dots, n_{y}\rbrace$\\
Iterates through the divisions in the $x$ direction\\
\textbf{for} $i=1,2,\dots,n_{x}$\\
\Indp Iterates through the divisions in the $y$ direction\\
\textbf{for} $j=1,2,\dots,n_{y}$\\
\Indp Generates a truss mesh with $i\times j$ divisions, \textit{i.e.} $\left(i+1\right)\left(j+1\right)$ nodes\\
Calculates the characteristic matrices $\mathbf{M}$ and $\mathbf{K}$ using FEA\\
\Indm \textbf{end}\\
\Indm \textbf{end}
\end{algorithmic}
\end{algorithm2e}

\section{Code}\label{sec:code}

\begin{ffcodeMimic}
sudo apt update
sudo apt install ca-certificates curl apt-transport-https
sudo install -m 0755 -d /etc/apt/keyrings
sudo curl -fsSL https://keys.anydesk.com/repos/DEB-GPG-KEY -o /etc/apt/keyrings/keys.anydesk.com.asc
sudo chmod a+r /etc/apt/keyrings/keys.anydesk.com.asc
\end{ffcodeMimic}

% Adds the appendix heading and sets the list of appendices
\AppendixSetup

\section{Appendix template}\label{sec:appendix_template}

% Inserts the bibliography using the bibliographia.bib file
\InsertBibliography{bibliographia}

\end{document}